
\documentclass{article}
\usepackage[english]{babel}
\usepackage[T1]{fontenc}
\frenchspacing
\author{KostiTeam}
\title{User Manual for LXC}
\begin{document}

\maketitle
\newpage
\tableofcontents
\newpage

\section{The Beginning}
All commands used are launched/done as root

\subsection{Install LXC}
\subsubsection{Arch-linux}
\emph{\#pacman -S lxc arch-install-scripts}
\subsubsection{Debian}
\emph{\#apt-get install lxc}

\subsection{Create a container}
\emph{\#lxc-create -t download -n cntName}: create a container which name is \emph{cntName}, 
by displaying a list, where the user can choose the OS he wants for the container.\\ 

\emph{\#lxc-create -t download -n cntName -- -d debian -r jessie -a i386}: create the \emph{cntName} container, 
by downloading an OS of \emph{debian} distribution, \emph{jessie} release, and \emph{i386} architecture
(32 bits).\footnote{create a container which has a 64 bits architecture on a 32 bits host computer \textbf{dosen't work}.}\\

\emph{\#lxc-ls}: display the list of all the created containers; --fancy for more details.\\
  
\subsection{Start a container}
\emph{\#lxc-start -n cntName  -d}: Start the \emph{cntName} container in daemon mode \emph{(-d)}.\\
By default, there isn't any user's account. So, it is necessary to connect as root.\\

\emph{\#lxc-attach -n cntName}: connect as root to the \emph{cntName} Container.\\
Once an user's account is created, and the container launched, it is possible to launch a terminal on a session\\

\emph{\#lxc-console -n cntName -t 0}: open a login screen on the \emph{tty0} terminal of the \emph{cntName} 
container.\footnote{\textbf{BUG:} on some OS, -t \textbf{must} be equal to 0}\\
Or, if the container is stopped, it is also possible to launch a container on a login screen by taking off the \emph{-d} to the lxc-start command.

\subsection{Configure a container}
\subsubsection{configuration file}
A container's configuration file can be found at /var/lib/lxc/\emph{<container name>}/config.\\
Here is a gateway configuration example:\\
\emph{RTFM: lxc.container.conf}\\

\noindent
\textbf{/var/lib/lxc/passerelle/config}\\

\noindent
\# Distribution configuration\\
lxc.include = /usr/share/lxc/config/debian.common.conf\\
lxc.arch = x86\_64\\

\noindent
\# Container specific configuration\\
lxc.rootfs = /var/lib/lxc/passerelle/rootfs\\
lxc.rootfs.backend = dir\\
lxc.utsname = passerelle\\

\noindent
\# Network configuration\\
lxc.network.type = veth\\
lxc.network.name = eth0\\
lxc.network.link = lxcbr0\\
lxc.network.flags = up\\
lxc.network.hwaddr = 00:16:3e:5b:0e:8f\\
lxc.network.ipv4 = 172.16.1.1\\
lxc.network.ipv6 = fec00:0:0:2::1\\

\noindent
lxc.network.type = veth\\
lxc.network.name = eth1\\
lxc.network.link = lxcbr0\\
lxc.network.flags = up\\
lxc.network.hwaddr = 00:16:3e:5b:0e:8f\\
lxc.network.ipv4 = 192.168.1.1/24\\
lxc.network.ipv6 = fc00:0:0:1::1/64\\

Here, the container has two interfaces (eth0 and eth1); each one has its own ipv4 (\emph{lxc.network.ipv4}), ipv6(\emph{lxc.network.ipv6}),
and MAC(\emph{lxc.network.hwaddr}) adresses.\\
\emph{lxc.network.flags} indicate what to do with the interface (\emph{up} activate the interface.).\\
\emph{lxc.network.type} indicate the type of network virtualisation to use (RTFM).\\
\emph{lxc.network.link} indicate the bridge to use for the package trafic between containers, this will be explained later.\\

\subsubsection{ifconfig, ip}

***IP commands/ifconfig for ipv4,ipv6***

\subsection{Join containers with bridges}

To join the continers to other containers, or to the host, LXC uses bridges. When a container is launched,
each "virtual" interfaces of the container will create an interface on the host. Join those interfaces
to a same bridge will allow to comunicate between severals interfaces (if there is a proper network configuration)

\subsubsection{The "manual" method}

After launching the containers, it is possible to connect them with the bridges:

\noindent
\emph{\#brctl addbr br0}: create a bridge which name is \emph{br0}\\
\emph{\#ifconfig br0 up}: activate the \emph{br0} interface\\
\emph{\#brctl addif br0 VETH12345}: Join the \emph{br0} bridge and the \emph{VETH12345} physical interface\footnote{See also the other doc to have more details}\\

\subsubsection{The automatic method}

It is also possible to join containers by the configuration file.\footnote{\textbf{Warning : }the bridge \textbf{must} be created manually, it can't be created with the config file.}
Be carefull, if you tried to launch a container, which depends on a bridge which is not created, it will return an error.

\noindent
\emph{\#lxc.network.link} allow an interface to connect to a specific bridge.\\

Exemple:\\
\noindent
lxc.network.type = veth\\
lxc.network.name = eth1\\
lxc.network.link = lxcbr0\\
lxc.network.flags = up\\
lxc.network.hwaddr = 00:16:3e:5b:0e:8f\\
lxc.network.ipv4 = 192.168.1.1\\
lxc.network.ipv6 = fc00:0:0:1::1\\

\noindent
the \emph{eth1} interface will be connected to the \emph{lxcbr0} bridge.


\end{document}
