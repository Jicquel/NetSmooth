%        File: cahier_des_charges.tex
%     Created: sam. oct. 22 06:00  2016 C
% Last Change: sam. oct. 22 06:00  2016 C
%
\documentclass[a4paper]{article}
\usepackage[francais]{babel}
\usepackage[T1]{fontenc}
\usepackage[utf8]{inputenc}
\usepackage{graphicx}
\usepackage{wrapfig}
\usepackage{lscape}
\usepackage{rotating}
\usepackage{epstopdf}
%\usepackage{kmath,kerkis,fourier}
%\selectfont
\setlength{\hoffset}{0pt}
\setlength{\oddsidemargin}{0pt}
\setlength{\footskip}{0pt}
\setlength{\textwidth}{455pt}


\frenchspacing
\author{KostiTeam}
\title{Cahier des charges}

\setcounter{secnumdepth}{4}

\begin{document}

\maketitle
\newpage
\tableofcontents
\newpage

\section{Probl\'ematique}
La virtualisation est le fait de cr\'eer une version virtuelle d'une entit\'e physique, ces versions virtuelles
sont alors appel\'ees Machines virtuelles ou VM (Virtual Machine). Nous pouvons prendre comme exemple Virtual
Box, qui est un outil de virtualisation permettant d'\'emuler un syst\`eme d'exploitation (linux ou autre).
Les diff\'erentes ressources de la machine h\^ote sont alors partag\'ees et allou\'ees dynamiquement aux
diff\'erentes machines virtuelles par des logiciels appel\'es hyperviseur.\\

La virtualisation peut \^etre utile dans diverses situations. En effet, virtualiser permet de tester un environnement afin de trouver ses meilleurs caractéristiques avant de le créer. De plus, la virtualisation peut être utilisée à des fins pédagogiques. Dans le cadre d'\'etudes informatiques, il peut
\^etre pratique de simuler des r\'eseaux dans lesquels des machines dialoguent. Marionnet est un exemple de
logiciel de simulation de r\'eseaux. 

\subsection {Marionnet}
Marionnet est un simulateur de réseau. Voici un exemple d'utilisation : 
On souhaite simuler un réseau constitué de 2 sous-réseaux. Chaque sous-réseau contient 3 machines.
On relie donc chaque groupe de 3 machines avec un hub, puis les deux sous-réseaux avec une passerelle (gateway).

On peut ensuite configurer les tables de routage, les adresses ipv4, ipv6 etc. pour simuler des échanges entre les machines.

Cependant, apr\`es avoir utiliser un tant soit peu Marionnet, il s'av\`ere que
plusieurs d\'efauts g\^enent son utilisation : 
\begin{itemize}
  \item crashs assez fr\'equents;
  \item impossible de changer les configurations des machines en marche;
  \item processus d'arr\^et des machines trop fastidieux;
  \item interface peu ergonomique;
  \item logiciel trop gourmand en ressources.
\end{itemize}


Face à ces probl\`emes, nous avons donc d\'ecid\'e de cr\'eer un nouveau simulateur de r\'eseaux.

\section{Solution}
Pour pallier les probl\`emes expos\'es pr\'ecedemment, il nous semble pertinent de ne pas virtualiser les
machines utilis\'ees  (VM) mais uniquement les environnements (VE).\\

Les VM ont un d\'efaut pour nous: elles simulent tout le materiel d'une machine (processeur, RAM...).
Nous prefererons donc utiliser des VE (Virtual Environement). Les VE permettent de ne simuler que le
systeme d'exploitation, et de partager le m\^eme noyeau que la machine h\^ote, et donc, de r\'epartir les
ressources entre machine h\^ote et les differents environnements virtuels. Cette solution répond donc
à nos besoins : un gain notable de performance, notamment lorsque le nombre de machines est
important.\\

Pour virtualiser les environnements, nous avons choisi d'utiliser la technologie de LXC.

\section{LXC}
\subsection{D\'efinition}

LXC est un outil de virtualisation permettant de cr\'eer des environnements virtuels, diff\'erents systemes
d'exploitations sont mis a disposition (Ubuntu, Debian...). Ces Environnements sont appel\'ees containers. Le
partage des ressources est assur\'e par l'outil Cgroups du noyau, qui permet de limiter, compter et
isoler l'utilisation des ressources.\\

Chaque environnement virtuel est isol\'ee, de la m\^eme mani\`ere que l'isolement d'un programme 
avec "\emph{chroot}": chaque environnement est cr\'e\'e de mani\`ere \`a ce qu'ils naient pas acc\`es au syst\`eme 
d'exploitation de la machine h\^ote. En revanche, la machine h\^ote, elle, a acc\`es aux machines virtuelles.
Cette isolation entre les machines virtuelles et la machine h\^ote, permet de garantir une certaine s\'ecurit\'e.\\

\subsection{Pourquoi LXC ?}

LXC poss\`ede sa propre API, notamment en C/C++, ce qui nous permet de l'int\'egrer ) notre projet efficacement.
Il est plus l\'eger que des technologies comme Docker (qui h\'erite de LXC) et convient \`a notre utilisation.

\subsection{Fonctionnement de LXC}

\subsubsection{Les containers, leur fonctionnement}

Les environnements virtuels, ou, containers (conteneurs en francais)  doivent, en premier temps, \^etre cr\'e\'es
\`a l'aide de la commande "\emph{lxc-create ...}" (voir notices pour plus de pr\'ecisions sur les commandes). De
nombreux syst\`emes d'exploitation seront alors propos\'es, (nous avons choisi de prendre Debian, Jessie,
i386). Par d\'efaut, les machines cr\'e\'ees ne sont pas configur\'ees: elles n'ont pas d'interfaces, elles 
n'ont pas de compilateur, et les utilitaires pr\'einstall\'es sont tr\`es rudimentaire 
(pas de ping, ifconfig, tcpdump...).\\

Chaque container poss\`ede un fichier de configuration situ\'e \`a l'emplacement suivant: "\emph{/var/lib/lxc/<nom 
du container>/config}". Il est possible de configurer de nombreux aspect du container dans ce fichier 
(par exemple: modifier les variables d'environnement, ou changer le nom d'h\^ote ("\emph{hostname}") du 
container. Voir "\emph{man lxc}" pour en savoir plus). Ce fichier va notamment permettre de param\'etrer la 
configuration r\'eseau du container \`a son d\'emarrage (c'est ce qui va nous int\'eresser en priorit\'e avec 
ce fichier).\\

Les containers se lancent avec la commande "\emph{lxc-start ...}"; il est pr\'ef\'erable de les lancer en d\'emons
(en arri\`ere-plan), puis d'y "\emph{attacher}" un terminal avec "\emph{lxc-attach}", afin d'\^etre connect\'e
en super utilisateur (ou root) car, premi\`erement, par d\'efaut, aucun profil d'utilisateur n'est cr\'e\'e sur 
le container, et, de plus, cela permet d'avoir l'entier contr\^ole du container afin de, par exemple, modifier 
son adresse ipv4.\\


\subsubsection{Bridges et fonctionnement r\'eseau}
\paragraph{fichier de configuration}~\\

Comme expliqu\'e plus haut, les param\`etres r\'eseaux du container peuvent \^etre modifi\'es via le fichier config.
Dans ce fichier, chaque "\emph{block}" permet de d\'efinir une interface. Un block commence toujours par la 
d\'efinition du type de r\'eseau, nous allons donc nous int\'eresser tout d'abord au type de r\'eseau
("\emph{lxc.network.type}").
Plusieurs types de r\'eseaux sont disponibles, mais, nous allons choisir le type veth, qui signifie Virtual 
Ethernet, ce type permet d'\'etablir un lien entre l'interface virtuelle du container et un pont, pr\'ealablement
cr\'e\'e sur la machine host (nous verrons cette liaison plus en d\'etail par la suite).\\

Ce fichier permet aussi d'\'etablir des adresses ipv4 ("\emph{lxc.network.ipv4}"), ipv6 ("\emph{lxc.network.ipv6}"),
et mac ("\emph{lxc.network.hwaddr}"), ainsi que leurs Broadcast. Cela permet de mettre en place le r\'eseau 
virtuel avant de lancer les machines; bien que ces param\`etres puissent \^etre modifi\'es une fois la machine
lanc\'ee \`a l'aide de l'outil "\emph{ifconfig}". Attention, les tables de routage ne peuvent pas \^etre
configur\'ees d'avance, il faut les configurer une fois le VE lanc\'e.\\

Le dernier param\`etre que nous verrons pour ce fichier est le lien ("\emph{lxc.network.link}"); ce param\`etre
va permettre d'indiquer \`a quel bridge nous voulons connecter notre interface.\\

\paragraph{Bridges}~\\

Les bridges (ou pont en francais) sont des \'equipements r\'eseaux qui permettent de relier deux (ou plus)
interfaces de mani\`ere compl\`etement transparente: en observant les paquets qui transitent, le pont peut 
connaitre les adresses mac des interfaces, et ainsi, rediriger les paquets. Les ponts peuvent par exemple,
\^etre utilis\'es pour rediriger une connexion Ethernet: une machine se connecte en Ethernet, une seconde 
machine se connecte \`a la premi\`ere, elles \'etablissent un pont entre deux de leurs interfaces (une interface
de la premi\`ere machine, et une interface de la deuxi\`eme machine), et, ainsi, la deuxi\`eme machine peut 
avoir acc\`es \`a internet.\\

Lorsqu'un VE se lance, une interface se cr\'e\'e sur la machine h\^ote pour chaque interface pr\'esente sur le VE.
Ces interfaces cr\'e\'ees ont un noms qui commencent toujours par "\emph{VETH}" suivit de quatre caract\`eres. Ces
interfaces sur la machine h\^ote representent les interfaces de l'environement virtuels, et vons nous permettre des
reliers les environnements entre eux; il est possible de voir la correspondance entre les interfaces d'un container
et les interfaces de la machine h\^ote grace a la commande "\emph{lxc-info ...}".\\

Il est donc necessaire de relier ces interfaces sur la machine h\^ote a un m\^eme bridge pour que les 
containers puissent communiquer!

\section{Langage et autres outils utilisés}
Les principaux langages utilisés seront : 
\begin{itemize}
  \item Le C++, car c'est un langage orienté objet compatible avec l'API C de LXC;
  \item Le langage shell pour utiliser des fonctions telles que la gestion des bridges (brctl).
\end{itemize}

Pour créer l'interface graphique, nous utiliserons l'API Qt. En effet, cette dernière nous semble être de qualité, et QtCreator rend la compilation plus pratique.

\newpage
\section{Architecture du programme}
\subsection{Vues}
Le logiciel a une seule interface : 
\begin{center}
\includegraphics[scale=0.25]{bulto.png}
\end{center}
Elle est divisible en 4 parties : 
\begin{itemize}
\item La barre supérieure (Fichier, Éditions, Aide...),
\item la barre de sélection (à gauche) où on peut choisir quel entité ajouter,
\item l'écran principal (au centre),
\item la barre d'information (à droite) où sont affichées les configurations des entités sélectionnées.
\end{itemize}

Voici un exemple avec un ping entre deux machines capturé avec tcpdump :
\vspace{16\baselineskip}
\begin{center}
\includegraphics[scale=0.25]{exemple_netsmooth.png}
\end{center}


\section{Etapes pr\'evues}
Voici un diagramme de Gantt qui décrit les étapes et l'avancement actuel du projet :\\
\includegraphics[scale=0.35]{gantt3.png}

\section{Notice d'utilisation de LXC}
\emph{Ce guide est destinée aux personnes voulant débuter avec LXC. Les bugs rencontrés sont précisés en bas de page. }

\subsection{Bien d\'ebuter}
Toutes les commandes ex\'ecut\'ees sont effectu\'ees en root (super-utilisateur).

\subsubsection{Installer LXC}
\paragraph{Arch-linux}
\emph{\#pacman -S lxc arch-install-scripts}
\paragraph{Debian}
\emph{\#apt-get install lxc}

\subsubsection{Cr\'eer un container}
\emph{\#lxc-create -t download -n name}: cr\'eer un container de nom name en proposant la liste des images d'OS possibles \`a t\'el\'echarger.\\ 

\emph{\#lxc-create -t download -n name -d debian -r jessie -a i386}: cr\'eer un container de nom \emph{name} en t\'el\'echargeant une image de distribution \emph{debian}, de release \emph{jessie} et d'architecture \emph{i386} (32 bits).\footnote{Cr\'eer un container d'architecture 64 bits sur un host 32 bits \textbf{ne fonctionne pas}.}\\

\emph{\#lxc-ls}: obtenir la liste des containers cr\'e\'es --fancy pour plus de d\'etails.\\
  
\subsubsection{D\'emarrer un container}
\emph{\#lxc-start -n name  -d}: d\'emarrer le container de nom \emph{name} en daemon \emph{(-d)}.\\
Par d\'efaut, aucun compte utilisateur n'est cr\'e\'e. Il faut donc se connecter en root.\\ 

\emph{\#lxc-attach -n name}: se connecter en root sur le container \emph{name}.\\
Une fois un compte utilisateur cr\'e\'e, et le container lanc\'e, il est possible de se connecter a une session en particulier.\\

\emph{\#lxc-console -n name -t 0}: ouvrir un \'ecran de login sur le terminal tty0 du container \emph{name}.\footnote{\textbf{BUG:} sur certaines distributions, -t diff\'erent de 0 ne fonctionne pas}\\
Ou, si le conatiner est stopp\'e, il est possible de lancer le container sur un \'ecran de login, en enlevant l'option \emph{-d} a la commande \emph{lxc-start}

\subsubsection{Configurer un container}
\paragraph{Fichier de configuration}
Le fichier de configuration d'un container \emph{name} est /var/lib/lxc/\emph{name}/config. Voici un exemple de configuration de passerelle:\\
\emph{RTFM: lxc.container.conf}\\

\noindent
\textbf{/var/lib/lxc/passerelle/config}\\

\noindent
\# Distribution configuration\\
lxc.include = /usr/share/lxc/config/debian.common.conf\\
lxc.arch = x86\_64\\

\noindent
\# Container specific configuration\\
lxc.rootfs = /var/lib/lxc/passerelle/rootfs\\
lxc.rootfs.backend = dir\\
lxc.utsname = passerelle\\

\noindent
\# Network configuration\\
lxc.network.type = veth\\
lxc.network.name = eth0\\
lxc.network.link = lxcbr0\\
lxc.network.flags = up\\
lxc.network.hwaddr = 00:16:3e:5b:0e:8f\\
lxc.network.ipv4 = 172.16.1.1\\
lxc.network.ipv6 = fec00:0:0:2::1\\

\noindent
lxc.network.type = veth\\
lxc.network.name = eth1\\
lxc.network.link = lxcbr0\\
lxc.network.flags = up\\
lxc.network.hwaddr = 00:16:3e:5b:0e:8f\\
lxc.network.ipv4 = 192.168.1.1\\
lxc.network.ipv6 = fc00:0:0:1::1\\

Ici, le container poss\`ede deux interfaces (eth0 et eth1) avec chacune leurs adresses ipv4 (\emph{lxc.network.ipv4}),
ipv6 (\emph{lxc.network.ipv6}) et MAC (\emph{lxc.network.hwaddr}).\\
lxc.network.flags indique quelle action effectuer (up active l'interface).\\
lxc.network.type indique quelle type de virtualisation de réseau utiliser (RTFM).\\
lxc.network.link indique l'interface \`a utiliser pour le vrai trafic, cette notion sera vu plus en d\'etail par la suite.\\

\paragraph{ifconfig, ip}

***Commandes ip/ifconfig pour ipv4,ipv6***

\subsubsection{Relier les containers avec les ponts (ou bridges)}

Pour relier les containers entre eux, ou m\^eme au host, LXC utilise les bridge (pont en francais). Lors du
lancement d'un container, l'activation de chacune de ses interfaces virtuelles va cr\'eer une interface sur la machine
host. Relier ces interfaces \`a un m\^eme pont permet de relier les containers "physiquement".

\paragraph{La m\'ethode manuelle}

Il est possible de relier les containers aux bridges apres les avoir lancer:

\noindent
\emph{\#brctl addbr br0}: cr\'ee un bridge de nom \emph{br0}\\
\emph{\#ifconfig br0 up}: active l'interface \emph{br0}\\
\emph{\#brctl addif br0 VETH12345}: relie le bridge \emph{br0} \`a l'interface physique \emph{VETH12345}\footnote{Voir l'autre doc pour plus de details sur le nom de l'interface}\\

\paragraph{La m\'ethode automatique}

Il est aussi possible de relier les containers par le fichier de configuration.\footnote{\textbf{Attention : }le pont ne peut pas \^etre \textbf{cr\'e\'e} par le fichier de configuration. Il faut le cr\'eer manuellement.}
Attention, si vous esssayez de lancer un container, qui d\'epend d'un bridge non cr\'e\'e, il y aura une erreur.

\noindent
\emph{\#lxc.network.link} permet de pr\'eciser le pont auquel relier l'interface cr\'e\'ee.\\

Exemple:\\
\noindent
lxc.network.type = veth\\
lxc.network.name = eth1\\
lxc.network.link = lxcbr0\\
lxc.network.flags = up\\
lxc.network.hwaddr = 00:16:3e:5b:0e:8f\\
lxc.network.ipv4 = 192.168.1.1\\
lxc.network.ipv6 = fc00:0:0:1::1\\

\noindent
L'interface eth1 sera reli\'ee (indirectement) au pont \emph{lxcbr0}.



\newpage
\section{Annexe}
\subsection{Diagrammes}

\subsubsection{Diagramme de cas d'utilisation :}

\begin{center}
\includegraphics[scale=0.47]{usecase_diagram.png}
\end{center}

\newpage
\subsubsection{Diagramme de classe d'analyse}
\begin{center}
\includegraphics[scale=0.5]{diagramme_analyse.png}
\end{center}

Les classes en rouge représentent les contrôleurs, les classe en vert les vues, puis les classes en bleu les modèles.

\newpage
\subsection{Diagrammes de séquence}
\textbf{Créer une machine, un hub ou une passerelle - première version}
\begin{center}
\includegraphics[scale=0.4]{createmachine.png}
\end{center}

\textbf{Démarrer une machine ou une passerelle} 
\begin{center}
\includegraphics[scale=0.55]{startmachine.png}
\end{center}

\vspace{17\baselineskip}
\textbf{Configurer une machine/passerelle éteinte} 
\begin{center}
\includegraphics[scale=0.4]{configuredownmachine.png}
\end{center}

\textbf{Ouvrir terminal d'une machine allumée} 
\begin{center}
\includegraphics[scale=0.6]{openterminal.png}
\end{center}

\vspace{16\baselineskip}
\textbf{Créer un cable} 
\begin{center}
\includegraphics[scale=0.4]{createcables.png}
\end{center}
\newpage
\subsection{Scénarios}

\subsubsection{Premier cas}

\noindent\textbf{TITRE}: Créer une machine virtuelle.\\
\textbf{OBJECTIF}: Pouvoir créer une machine pour l’utiliser dans un réseau.\\
\textbf{ACTEURS}: L’utilisateur.\\
\textbf{TYPE}: Scénario nominale.\\
\textbf{PRECONDITIONS}: Il faut avoir lancé le logiciel.\\
\textbf{POSTCONDITIONS}: La machine est créée.\\
\textbf{DESCRIPTIF}:\\ 
\indent1. L’utilisateur clique sur l’icône de la machine.\\
\indent2. L’utilisateur déplace l’icône sur la zone qui représente le réseau.\\
\indent3. L’utilisateur saisit l’adresse IP4 et/ou IP6 de la machine.\\
\indent4. L’utilisateur saisit le masque 4 et/ou 6 de la machine.\\
\indent5. L’utilisateur veut définir des paramètres de routages.\\
\indent6. L’utilisateur saisit l’adresse destination en IP4/IP6.\\	
\indent7. L'utilisateur saisit l'adresse de la passerelle en IP4/IP6\\
\indent8. L’utilisateur saisi l’interface qu’il veut utiliser.\\
\textbf{FLUX ALTERNATIFS:}
\indent5. L'utilisateur ne veut pas définir des paramètres de routage.\\
\indent\indent5a. L'utilisateur laisse les champs vides.\\
\textbf{CAS REFERENCES}: aucun.\\

\subsubsection{Deuxième cas}
\noindent\textbf{TITRE}: Créer une passerelle.\\
\textbf{OBJECTIF}: Pouvoir créer une passerelle et l’utiliser dans le réseau.\\
\textbf{ACTEURS}: L’utilisateur.\\
\textbf{TYPE}: Scénario nominale.\\
\textbf{PRECONDITIONS}: Avoir lancé le logiciel.\\
\textbf{POSTCONDITIONS}: La passerelle est créée.\\
\textbf{DESCRIPTIF}:\\
\indent1. L’utilisateur clique sur l’icône de la passerelle.\\
\indent2. L’utilisateur déplace l’icône sur la zone qui représente le réseau.\\
\indent3. L’utilisateur choisi le nombre d’interfaces qu’il veut.\\
\indent4. L’utilisateur saisi l’adresse IP4 et/ou IP6 de la passerelle pour toutes ses interfaces.\\
\indent5. L’utilisateur saisi le masque 4 et/ou 6 de la passerelle pour toutes ses interfaces.\\
\indent6. L’utilisateur veut définir des paramètres de routages.\\
\indent7. L’utilisateur saisi l’adresse destination en IP4/IP6.\\
\indent8. L’utilisateur saisi l’adresse de la passerelle en IP4/IP6.\\
\indent9. L’utilisateur saisi l’interface qu’il veut utiliser.\\
\textbf{FLUX ALTERNATIFS}:\\
\indent5. L'utilisateur ne veut pas définir des paramètres de routage.\\
\indent\indent5a. L'utilisateur laisse les champs vides.\\
\textbf{CAS REFERENCES}: aucun.\\


\subsubsection{Troisième cas}
\noindent\textbf{TITRE}: Créer un hub.\\
\textbf{OBJECTIF}: Pouvoir créer un hub pour l’utiliser dans un réseau.\\
\textbf{ACTEURS}: L’utilisateur.\\
\textbf{TYPE}: Scénario nominale.\\
\textbf{PRECONDITIONS}: Il faut avoir lancé le logiciel.\\
\textbf{POSTCONDITIONS}: Le hub est créé.\\
\textbf{DESCRIPTIF}: 
\indent1. L’utilisateur clique sur l’icône du hub.\\
\indent2. L’utilisateur déplace l’icône sur la zone qui représente le réseau.\\
\textbf{CAS REFERENCES}: aucun.\\



\subsubsection{Quatrième cas}
\noindent\textbf{TITRE}: Relier une machine à une machine/hub/passerelle.\\
\textbf{OBJECTIF}: Pouvoir relier une machine avec un autre dispositif pour qu’il puisse se « voir» dans le réseau.\\
\textbf{ACTEURS}: L’utilisateur.\\
\textbf{TYPE}: Scénario nominale.\\
\textbf{PRECONDITIONS}: Il faut avoir lancé le logiciel, créée une machine et un autre dispositif.\\
\textbf{POSTCONDITIONS}: La machine est reliée avec l’autre dispositif.\\
\textbf{DESCRIPTIF}: \\
\indent1. L’utilisateur clique sur l’icône du câble.\\
\indent2. L’utilisateur déplace l’icône sur la machine à relié.\\
\indent3. L’utilisateur clique sur le dispositif auquel il veut relier la machine.\\
\textbf{CAS REFERENCES}: aucun.\\

\subsubsection{Cinquième cas}
\noindent\textbf{TITRE}: Allumer un dispositif.\\
\textbf{OBJECTIF}: Pouvoir allumer un dispositif pour qu’il soit visible sur le réseau.\\
\textbf{ACTEURS}: L’utilisateur.\\
\textbf{TYPE}: Scénario nominale.\\
\textbf{PRECONDITIONS}: Il faut avoir lancé le logiciel, et créer un dispositif.\\
\textbf{POSTCONDITIONS}: Le dispositif est allumé.\\
\textbf{DESCRIPTIF}:\\
\indent1. L’utilisateur clique sur l’icône on/off.\\
\textbf{CAS REFERENCES}: aucun.\\

\end{document}


